\documentclass[12pt]{exam}
\usepackage{amsthm}
\usepackage{libertine}
\usepackage[utf8]{inputenc}
\usepackage[margin=1in]{geometry}
\usepackage{amsmath,amssymb}
\usepackage{multicol}
\usepackage[shortlabels]{enumitem}
\usepackage{siunitx}
\usepackage{cancel}
\usepackage{graphicx}
\usepackage{adjustbox}
\usepackage{pgfplots}
\usepackage{listings}
% \usepackage{tikz} %Tikz is a language used to vector diagrams. Import when needed.


%references
\usepackage[sort&compress,square,numbers]{natbib}
% formats URL nicely, especially in the bibliography
\usepackage{url}


\pgfplotsset{width=10cm,compat=1.9}
\usepgfplotslibrary{external}
\tikzexternalize



% -------------------------------------------------------
% -------------------------------------------------------
% -------------------------------------------------------
% -------------------------------------------------------
% ---------------------UPDATE DETAILS--------------------
\newcommand{\class}{\MakeUppercase{}} %SUBJECT
\newcommand{\name}{Utkarsh Sethi} % YOUR NAME
\newcommand{\examnum}{} % ASSIGNMENT NUMBER
\newcommand{\examdate}{\today} % This is the due date e.g. August 22, 2022
% \newcommand{\timelimit}{}
% -------------------------------------------------------
% -------------------------------------------------------
% -------------------------------------------------------
% -------------------------------------------------------
% -------------------------------------------------------




\begin{document}
\pagestyle{plain}
\thispagestyle{empty}

\noindent
\begin{tabular*}{\textwidth}{l @{\extracolsep{\fill}} r @{\extracolsep{6pt}} l}
    \textbf{Name:} {\name~(7336330)} %Your name
    & \class\\
    \examdate %Comment to not include
    & \examnum\\
\end{tabular*}\\
\rule[2ex]{\textwidth}{2pt}
% ---




\begin{enumerate}[label=Ans \arabic*:, leftmargin=*] %You can make lists!
    
    
    %%Template
    % \item{
    
    % }
    
    
    %1
    \item{
          
          }
          
          %2
    \item{
          
          }
          
          %3
    \item{
          
          }
          
          %4
    \item{
          
          }
          
          %5
    \item{
          
          }
          
          %6
    \item{
          
          }
          
          %7
    \item{
          
          }
          
          %8
    \item{
          
          }
          
          %9
    \item{
          
          }
          
          %10
    \item{
          
          }
          
\end{enumerate}


%References
\bibliographystyle{IEEEtranN}
\bibliography{bibliography}
%create a file named 'bibliography.bib'

\end{document}
